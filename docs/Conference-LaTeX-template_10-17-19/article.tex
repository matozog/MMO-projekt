\documentclass[conference]{IEEEtran}
\IEEEoverridecommandlockouts
% The preceding line is only needed to identify funding in the first footnote. If that is unneeded, please comment it out.
\usepackage{cite}
\usepackage{amsmath,amssymb,amsfonts}
\usepackage{algorithmic}
\usepackage{graphicx}
\usepackage{textcomp}
\usepackage{polski}
\usepackage{xcolor}
\def\BibTeX{{\rm B\kern-.05em{\sc i\kern-.025em b}\kern-.08em
    T\kern-.1667em\lower.7ex\hbox{E}\kern-.125emX}}
\begin{document}

\title{Porównanie wydajności algorytmów drzewiastych w problemie klasyfikacyjnym wykorzystującym niezbalansowany zbiór danych dla choroby Parkinsona}

\author{Mateusz Ozóg, Mikołaj Mazurek}

\IEEEaftertitletext{\begin{center}
Październik 25, 2019 \\
\end{center}}
\maketitle

\begin{abstract}
Dokument ten przedstawia porównanie algorytmów drzewiastych w problemie klasyfikacyjnym związanym z detekcją choroby Parkinsona u osób starszych. W niniejszym artykule przedstawione zostaną klasyfikator tj.: Random Forest, Reduced Error Pruning Tree (REPTree) raz Random Tree. W celu porównania algorytmów przeprowadzone zostały testy statystyczne oraz wyciągnięto na ich podstawie odpowiednie wnioski. Do przeprowadzenia badań wykorzystano implementacje klasyfikatorów z programu WEKA (Waikato Environment for Knowledge Analysis)\footnote{https://www.cs.waikato.ac.nz/ml/index.html}.
\end{abstract}

\begin{IEEEkeywords}
classification, imbalance, uci repository, random forest
\end{IEEEkeywords}

\section{Wprowadzenie}


\section{Powiązane prace}
	W literaturze naukowej możemy znaleźć bardzo wiele artykułów podejmujących próby rozwiązywania problemów klasyfikacyjnych przy pomocy algorytmów drzewiastych m.inn. użytych w niniejszej pracy. Efektywne przetwarzanie i analiza danych są aktualnie bardzo modne, co powoduje, że wykorzystywane są w różnorakich dziedzinach życia i nauki.\\ 
\indent	Jedną z takich prac podejmującą klasyfikację w dziedzinie astronomii jest artykuł Yongheng Zhoa i Yanxia Zhanga, który porównuje klasyfikatory wykorzystujące schemat drzewa w celu wyszukiwania aktywnych obiektów w kosmosie. Badania odbyły się na podstawie niezbalansowanego zbioru danych zawierającego m.inn.: 3718 gwiazd, 173 normalnych galaktyk, 909 kwazarów, 612 aktywnych galaktyk optycznych itd. Do klasyfikacji wykorzystano następujące algorytmy: REPTree, Random Tree, Decision Stump, Random Forest, J48, NBTree i AdTree. Do przeprowadzenia badań zostały wykorzystane implementacje z programu WEKA wraz z domyślnie ustawionymi parametrami. Najlepszy wynik osiągnął klasyfikator ADTree oraz NBTree (ponad 97\%). W kontekście niniejszej pracy oraz porównywanych w niej algorytmów najlepiej wypadł Random Forest (około 97\%), trochę gorzej Random Tree - 87\% oraz REPTree - około 80\%. Jednak porównując złożoności czasowe RF był najwolniejszy, nawet 10-krotnie w stosunku do pozostałych dwóch klasyfikatorów. \cite{b1} \\
\indent Kolejnym przykładem 
	
 

\section{Opis zastosowanych klasyfikatorów}

\section{Implementacja}

\section{Przeprowadzone badania}

\section{Testy statystyczne oraz porównanie wyników}

\section{Konkluzja}

\begin{thebibliography}{00}
\bibitem{b1} Yongheng Zhao and Yanxia Zhang, ``Comparison of decision tree methods for finding active objects'' National Astronomical Observatories, CAS, 20A Datun Road, Chaoyang District, Bejing 100012 China.
\bibitem{b2} Sushilkumar Kalmegh, ``Analysis of WEKA Data Mining Algorithm REPTree, Simple Cart and RandomTree for Classification of Indian News'', associate Professor, Department of Computer Science, Sant Gadge Baba Amravati University
Amravati, Maharashtra- 444602, India.
\bibitem{b3} Zahra Nematzadeh Balagatabi, Roliana Ibrahim and Hossein Nematzadeh Balagatabi, ``Comparison of Decision Tree Methods in Classification of Researcher’s Cognitive styles in Academic Environment'', Faculty of Computer Science and Information Systems, Universiti Teknologi Malaysia, 81310 UTM Johor Bahru, Johor, Malaysia.
\bibitem{b4} Mohmad Badr Al Snousy, Hesham Mohamed El-Deeb, Khaled Badran and Ibrahim Ali Al Khlil, ``Suite of decision tree-based classification algorithms on cancer gene expression data'' Department of Computer Science, Sadat Academy for Management Science (SAMS), Modern University for Technology and Information (M.T.I.),  Military Technical College, Egypt, 23 July 2011.

\bibitem{b5} R. Nicole, ``Title of paper with only first word capitalized,'' J. Name Stand. Abbrev., in press.
\bibitem{b6} Y. Yorozu, M. Hirano, K. Oka, and Y. Tagawa, ``Electron spectroscopy studies on magneto-optical media and plastic substrate interface,'' IEEE Transl. J. Magn. Japan, vol. 2, pp. 740--741, August 1987 [Digests 9th Annual Conf. Magnetics Japan, p. 301, 1982].
\bibitem{b7} M. Young, The Technical Writer's Handbook. Mill Valley, CA: University Science, 1989.
\end{thebibliography}

\end{document}
